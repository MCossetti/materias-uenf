% Prof. Dr. Ausberto S. Castro Vera
% UENF - CCT - LCMAT - Curso de Ci\^{e}ncia da Computa\c{c}\~{a}o
% Campos, RJ,  2023
% Disciplina: Paradigmas de Linguagens de Programa\c{c}\~{a}o
% Aluno:




\noindent
\textbf{Disciplina:} \textit{Paradigmas de Linguagens de Programa\c{c}\~{a}o 2023}\\
\textbf{Linguagem:} \textit{Python}\\
\textbf{Aluno:} \textit{ \color{blue} Mariana Cossetti Dalfior}


\section*{Ficha de avalia\c{c}\~{a}o:}



\begin{tabular}{|p{12cm}|c|}
  \hline
  % after \\: \hline or \cline{col1-col2} \cline{col3-col4} ...
  \textbf{Aspectos de avalia\c{c}\~{a}o (requisitos m\'{\i}nimos)} & \textbf{Pontos} \\
  \hline
   \color{red} Introdu\c{c}\~{a}o (M\'{a}ximo: 01 pontos) &  \\
  $\bullet$ Aspectos hist\'{o}ricos &  \\
  $\bullet$ \'{A}reas de Aplica\c{c}\~{a}o da linguagem &  \\
  \hline
 \color{red}  Elementos b\'{a}sicos da linguagem (M\'{a}ximo: 01 pontos) &  \\
  $\bullet$ Sintaxe (vari\'{a}veis, constantes, comandos, opera\c{c}\~{o}es, etc.) &  \\
  $\bullet$ Cada elemento com exemplos (c\'{o}digo e execu\c{c}\~{a}o) &  \\
  \hline
  \color{red} Aspectos Avan\c{c}ados da linguagem (M\'{a}ximo: 2,0 pontos) &  \\
  $\bullet$ Sintaxe (vari\'{a}veis, constantes, comandos, opera\c{c}\~{o}es, etc.) &  \\
  $\bullet$ Cada elemento com exemplos (c\'{o}digo e execu\c{c}\~{a}o) &  \\
  $\bullet$ Exemplos com fonte diferenciada (listing) & \\
  \hline
  \color{red} M\'{\i}nimo 5 Aplica\c{c}\~{o}es completas - Aplica\c{c}\~{o}es (M\'{a}ximo : 2,0 pontos) &  \\
  $\bullet$ Uso de rotinas-fun\c{c}\~{o}es-procedimentos, E/S formatadas &  \\
  $\bullet$ Uma Calculadora &  \\
  $\bullet$ Gr\'{a}ficos &  \\
  $\bullet$ Algoritmo QuickSort &  \\
  $\bullet$ Outra aplica\c{c}\~{a}o &  \\
  $\bullet$ Outras aplica\c{c}\~{o}es ... &  \\
  \hline
  \color{red} Ferramentas (compiladores, interpretadores, etc.) (M\'{a}ximo : 1,0 pontos) &  \\
  $\bullet$ Ferramentas utilizadas nos exemplos: pelo menos DUAS&  \\
  $\bullet$ Descri\c{c}\~{a}o de Ferramentas existentes:  m\'{a}ximo 5&  \\
  $\bullet$ Mostrar as telas dos exemplos junto ao compilador-interpretador&  \\
  $\bullet$ Mostrar as telas dos resultados com o uso das ferramentas &  \\
  $\bullet$ Descri\c{c}\~{a}o das ferramentas (autor, vers\~{a}o, homepage, tipo, etc.) &  \\
  \hline
  \color{red} Organiza\c{c}\~{a}o do trabalho (M\'{a}ximo: 01 ponto) &  \\
  $\bullet$ Conte\'{u}do, Historia, Se\c{c}\~{o}es, gr\'{a}ficos, exemplos, conclus\~{o}es, bibliografia &  \\
  $\bullet$ Cada elemento com exemplos (c\'{o}digo e execu\c{c}\~{a}o, ferramenta, nome do aluno) &  \\
  \hline
  \color{red} Uso de Bibliografia (M\'{a}ximo: 01 ponto)&  \\
   $\bullet$ Livros: pelo menos 3&  \\
   $\bullet$ Artigos cient\'{\i}ficos: pelo menos 3 (IEEE Xplore, ACM Library)&  \\
   $\bullet$ Todas as Refer\^{e}ncias dentro do texto, tipo [ABC 04] & \\
   $\bullet$ Evite Refer\^{e}ncias da Internet & \\
   \hline
     &  \\
  \color{red} Conceito do Professor (Opcional: 01 ponto) & \\
  \hline
   & \\
  \hfill \color{blue} Nota Final do trabalho: & \\
  \hline
\end{tabular}\\
\textit{Observa\c{c}\~{a}o:} Requisitos m\'{\i}nimos significa a \textit{metade} dos pontos
